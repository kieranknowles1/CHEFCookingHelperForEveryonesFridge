\documentclass[../CHEFCookingHelperForEveryonesFridge.tex]{subfiles}

\ifSubfilesClassLoaded{\raggedbottom}{}

\renewcommand{\thesubsection}{\arabic{subsection}.}

\begin{document}

\newcommand{\torrevision}{1.0}

\begin{table}[h!]
    \centering
    \begin{tabular}{|c|c|}
        \hline
        \multicolumn{2}{|c|}{KV6003: Individual Computing Project Terms of References revision \torrevision{}} \\\hline
        Student name: & Kieran Knowles \\\hline
        Student ID: & w20013000 \\\hline
        Course: & BSC (HONS) Computer Science \\\hline
        Project Title: & \chef{} - Cooking Helper for Everyone's Fridge \\\hline
        Supervisor: & Nick Dalton \\\hline
        2\textsuperscript{nd} Marker: & Marta Cecchinato \\\hline
    \end{tabular}
\end{table}

\subsection{Background to Project}
As of 2020, UK households collectively wasted 4.5 million tons of food every year, or nearly 20\% of all purchased food.
Whilst aware that food waste is increasing among consumers, \textquote*{many do not yet acknowledge that this is an issue relevant to them or are not
yet concerned enough to act} \cite{wrap_uk_2020}.

With the current cost of living crisis, where inflation, especially in food costs, is vastly outpacing wage growth \cite{office_for_national_statistics_average_2023}
\cite{francis-devine_rising_2023}, families are looking to cut costs wherever possible. One way to do this is to reduce food waste which is a
primary aim of this project.

One possible reason for this is consumers not knowing what to do with the food they have in their fridge, and therefore not using it before it goes off.
Manually looking through a recipe book and checking that you have all the ingredients is a time-consuming process, and therefore people are less likely to do it.

This project aims to create a web application that tracks what ingredients the user has and then suggests recipes that can be made with those ingredients.
This will help to reduce food waste by helping users work with what they already have rather than buying more food and letting what they have go off.

The project is of interest to me personally as I often struggle to decide what to eat and instead end up eating fast food,
the same meals over and over, or getting ready meals from the shop.

The idea for the project came due to the increase in cost of living, and the realization of how much food we bought that was
never consumed. I was searching for recipes that incorporated foods I had. This was time-consuming and laborious.
I had the idea to create an app where I could simply type in what food I had and would return how that could be incorporated
into a meal.

\subsection{Proposed Work}
The project will be the development of \chef{}, standing for Cooking Helper for Everyone's Fridge. I have chosen this name
as it is a catchy and memorable acronym that describes the purpose of the project, helping people to make use of
what they already have in their fridge.

\chef{} will be a web application intended to be used on a mobile phone. It will allow users to create an account and log in,
then add the ingredients that they have either by manually entering them or by scanning the barcode on the packaging.
This ingredient will then be added to the user's virtual fridge and the user will be able to see what ingredients they have and how much of each.
If a barcode is not found in the database, the user can manually enter the type of ingredient and its quantity (weight, volume, or quantity depending on the ingredient).
This information will then be stored in the database for future use. As \chef{} will likely use an open-source dataset for barcodes, it could provide an option
for users to submit their new barcodes to the dataset to help improve it.

Whole foods such as fruits and vegetables will be entered by the user manually, as they do not have barcodes. This will likely use an interface where
all supported ingredients are shown and the user clicks to add one to their fridge, clicking multiple times to add multiple of the same ingredient.
Another option would be to use a classifier to identify the type of fruit or vegetable from a picture, however, this may be less accurate and more
time-consuming than the manual method.

\chef{} will track what kinds of recipes each user has made recently and avoid suggesting similar recipes
to ones that the user has recently made, such as different pasta dishes. This will help to avoid users
getting bored of eating the same meals over and over.

Users within the same household will be able to invite each other to an in-app household that shares a fridge.
This will allow multiple people to add ingredients and use them to cook meals without having to sync up
what they have.

If a user does not like a certain ingredient, such as mushrooms, they can specify this in their profile and
the \chef{} app will not suggest recipes including it. This is necessary as other members of the household
may like the ingredient and therefore add it to the shared fridge.

When finding a recipe, users will be able to select which members of the household they are cooking for.
\chef{} will then find recipes that everyone likes in addition to using the available ingredients.

A competitor analysis will be performed to identify features that are common among similar apps. The user reviews of existing apps
will be consulted to identify features that users like and those that are commonly requested. This will help to identify
improvements that could be made for \chef{}. The competitor analysis has been started in the \mainfilehyperref{sec:competitor_analysis}{Competitor Analysis}
section of the main report with the apps \cite{myfridgefood_myfridgefood_nodate} and \cite{supercook_supercook_nodate} which has identified the following
requirements for the \chef{} app which will be expanded upon in the main report:

\subsubsection{Must Have}
\begin{itemize}
    \item The system must allow for multiple users to share a virtual fridge.
    \item The system must allow for dietary requirements (e.g., allergies) to be specified for each user and must not suggest recipes that do not meet these requirements.
    \item The system must allow users to specify disliked ingredients that will not be used in suggested recipes.
    \item The system must allow users to submit new recipes, either to be shared with all users or kept private.
\end{itemize}

\subsubsection{Should Have}
\begin{itemize}
    \item The system should allow users to scan ingredients into their virtual fridge.
    \item The system should track how much of each ingredient is available.
    \item The system should deduct from the available ingredients when a recipe is made.
    \item The system should track which recipes have been made recently and avoid suggesting similar recipes.
    \item The system should allow users to favorite recipes and have similar recipes suggested.
    \item All ingredient lists should use metric units by default.
\end{itemize}

\subsubsection{Could Have}
\begin{itemize}
    \item The system could allow for the scanning of a receipt to automatically add ingredients to the virtual fridge.
    \item The system could provide an estimate of the cost of each recipe.
    \item The system could provide an estimate of how much waste/money has been saved by using it.
    \item The system could allow users to contribute to the datasets accessed by the app.
    \item The system could use machine learning to group similar recipes together and suggest items similar to what the user has made before.
\end{itemize}

The technologies required for the \chef{} app are discussed below in the \hyperref[sec:resources_statement]{Resources} section.

\subsection{Aims of Project}

\begin{enumerate}
    \item To create a system to reduce food waste by helping users make use of ingredients they already have.
    \item To encourage users to try new recipes that they may not have known they could make.
\end{enumerate}

\subsection{Objectives}

\begin{enumerate}
    \item Research existing solutions to identify what features are useful and what can be improved upon.
    \item Research potential data sources for ingredients and recipes.
    \item Research potential technologies for identifying ingredients from barcodes, images, and receipts.
    \item Expand on the existing requirements to create a full set of requirements for the \chef{} app.
    \item Develop the \chef{} application in line with the requirements.
    \item Test the \chef{} application to ensure that it meets the requirements.
    \item Evaluate the final product to determine if it meets the aims of the project.
\end{enumerate}

\subsection{Skills}
The following skills will be used in the project:

\begin{itemize}
    \item Programming skills learned in \textit{KV4000 Programming 1}, \textit{KV4001 Programming 2},
          \textit{KF5012 Software Engineering Practice}, and \textit{KF5008 Program Design and Development}.
    \item Web development skills learned in \textit{KF4009 Web Technologies}, \textit{KF5002 Web Programming},
          and \textit{KF6012 Web Application Integration}.
    \item Web API development learned in \textit{KF6012 Web Application Integration}.
    \item Database programming learned in \textit{KC4000 Relational Databases} and \textit{KF5008 Program Design and Development}.
    \item Project planning skills learned in \textit{KF4011 Systems Analysis}
    \item Interface design skills learned in \textit{KV5003 Human-Computer Interaction}.
    \item Machine learning to categorize similar items learned in \textit{KF5042 Intelligent Systems}.
    \item Container-based server deployment, to be learned from the \textit{Docker} documentation.
\end{itemize}

\ifSubfilesClassLoaded{
    % Print the bibliography on the same page
    % From bib latex docs
    \ifSubfilesClassLoaded{
        \defbibheading{bibliography}[\bibname]{
            \subsection{#1}
            \markboth{#1}{#1}
        }
    }{}
    \printbibliography
}{
    \subsection{References}
    See the \hyperref[sec:reference_list]{References} section of the main report for the main reference list, including those that were consulted when preparing the TOR.
}

\subsection{Resources - Statement of Hardware / Software Required}
\label{sec:resources_statement}

\chef{} is intended to use open-source software wherever possible and to be deployed on a local server which I already own.

A setup script will be created to install all dependencies and deploy the server. This will allow for quick and easy deployment to a new server
if required and for a reproducible development environment.

Recipes and barcodes will be sourced from an external API, preferably open-source. This will remove the need to manually enter data and allow for
\chef{} to automatically update with new recipes and barcodes as its data sources are updated.

If needed, more demanding tasks such as image recognition will be offloaded to a cloud service such as Azure as the server I own has limited hardware.

Both the front and backend code will be written in TypeScript with the frontend using the React framework and the backend using Node.js.

\subsection{Assessment Criteria for Practical Computing Work}

\begin{table}[h!]
    \begin{tabulary}{\textwidth}{|L|l|}
        \hline
        \textbf{Criteria} & \textbf{Mark} \\\hline
        \makecell[l]{
            Fitness for Purpose \\
            - How well the app meets the requirements \\
        } & 40 \\\hline
        \makecell[l]{
            Build Quality \\
            - Whether the app passes its unit tests \\
            - Quality of documentation \\
            - Whether the app runs without issue \\
            - Ease of setup \\
        } & 40 \\\hline
        \makecell[l]{
            Usability \\
            - Responsive layout on different devices \\
            - Performance - the app should feel responsive to use \\
        } & 20 \\\hline
    \end{tabulary}
\end{table}

\subsection{Ethics, Social, Legal, and Professional Issues}

\ifSubfilesClassLoaded{
    \subfile{EthicsApprovalForm.tex}
}{
    See the \hyperref[sec:ethics_approval]{Ethics Approval Form} appendix for the ethics approval form.
}

\subsection{Project Plan - Schedule of Activities}

\newganttlinktype{rd}{
    \ganttsetstartanchor{on right=0.5}
    \ganttsetendanchor{on top=0.5}
    \draw[/pgfgantt/link]
        (\xLeft, \yUpper) --
        (\xRight, \yUpper) --
        (\xRight, \yLower);
}

\begin{ganttchart}[hgrid, vgrid]{6}{12}
    \gantttitle{Week 6 to Christmas}{7} \\
    \gantttitlelist{6,...,12}{1} \\

    \ganttbar[name=competitor_analysis]{Competitor Analysis}{6}{6} \\
    \ganttbar[name=find_data_sources]{Find Data Sources}{6}{7} \\
    \ganttbar[name=research_ingredient_scan]{Research Ingredint Scanning Methods}{6}{7} \\

    \ganttbar[name=requirements]{Complete Requirements Specification}{8}{9} \\
    \ganttlink[link type=rd]{competitor_analysis}{requirements}
    \ganttlink[link type=rd]{find_data_sources}{requirements}
    \ganttlink[link type=rd]{research_ingredient_scan}{requirements}

    \ganttmilestone[name=implementation]{Start Implementation}{9} \\
    \ganttlink[link type=dr]{requirements}{implementation}

    \ganttbar[name=track_ingredients]{Track Ingredients}{10}{10} \\
    \ganttlink[link type=rd]{implementation}{track_ingredients}

    \ganttbar[name=suggest_recipes]{Suggest Recipes}{11}{12}
    \ganttlink[link type=dr]{track_ingredients}{suggest_recipes}
\end{ganttchart}

\begin{ganttchart}[hgrid, vgrid]{1}{16}
    \gantttitle{January to Submission}{16} \\
    \gantttitlelist{1,...,16}{1} \\

    \ganttbar[name=diet]{Specify Dietary Requirements}{3}{3} \\

    \ganttbar[name=scan]{Scan Ingredients Into the App}{4}{4} \\
    \ganttlink[link type=dr]{diet}{scan}

    \ganttbar[name=cost]{Estimate the Cost of Recipes}{5}{5} \\
    \ganttlink[link type=dr]{scan}{cost}

    \ganttbar[name=test]{Test the final \chef{} App}{6}{6} \\
    \ganttlink[link type=dr]{cost}{test}

    \ganttbar[name=report]{Prepare Report}{6}{14}
    \ganttlink[link type=dr]{cost}{report}
\end{ganttchart}

\end{document}
