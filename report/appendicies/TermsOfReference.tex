\documentclass[../CHEFCookingHelperForEveryonesFridge.tex]{subfiles}

\usepackage[style=authoryear]{biblatex}
\usepackage{csquotes}
\addbibresource{TermsOfReference.bib}

\renewcommand{\cite}[1]{\parencite{#1}}

\renewcommand{\thesubsection}{\arabic{subsection}.}

% Print the bibliography on the same page
% From biblatex docs
\defbibheading{bibliography}[\bibname]{
    \subsection{#1}
    \markboth{#1}{#1}
}

\begin{document}

\begin{table}[h!]
    \centering
    \begin{tabular}{|c|c|}
        \hline
        \multicolumn{2}{|c|}{KV6003: Individual Computing Project Terms of References} \\\hline
        Student name: & Kieran Knowles \\\hline
        Student ID: & w20013000 \\\hline
        Course: & BSC (HONS) Computer Science \\\hline
        Project Title: & \chef{} - Cooking Helper for Everyone's Fridge \\\hline
        Supervisor: & \todo{Supervisor} \\\hline
        2\textsuperscript{nd} Marker: & \todo{2nd marker} \\\hline
    \end{tabular}
\end{table}

\subsection{Background to Project}
\todoinline{
    This should describe the “context” of the proposed project and answer the question, “Why this project?” both from your perspective as a student undertaking
    a final year computing project and that of any client. It should show what makes this proposal a worthwhile computing final-year project.
    It must make clear both the application area or area of investigation and the computing aspects of your work.
    It will be helpful to start by briefly stating the nature of the product that you intend to produce or the research question that you intend to investigate.
}
As of 2020, UK households collectively wasted 4.5 million tons of food every year, or nearly \percentage{20} of all purchased food.
Whilst aware that food waste is increasing among consumers, "many do not yet acknowledge that this is an issue relevant to them or are not
yet concerned enough to act" \cite{wrap_uk_2020}.
\todo{Mention cost of living crisis}

One possible reason for this is consumers not knowing what to do with the food they have in their fridge, and therefore not using it before it goes off.
Manually looking through a recipe book and checking that you have all the ingredients is a time-consuming process, and therefore people are less likely to do it.

\todoinline{
    After that, describe the problem domain you will be working in. This may include a practical context - for example, a client organisation and its needs.
    It will certainly involve introducing the field in which you will be working. For example, what previous work suggested your research question?
}
This project aims to create a web application that tracks what ingredients the user has and then suggests recipes that can be made with those ingredients.
This will help to reduce food waste by helping users work with what they already have rather than buying more food and letting what they have go off.

\todoinline{
    What ideas will be important to defining and carrying out your project?  For software engineering projects in particular, it will usually be appropriate to introduce
    the general features of the kind of application you are trying to build, and ideas related to the problem that you will investigate in your literature review.
    You should by now have done a fair amount of reading about the area, to put your ideas into context and to ensure that the project idea is valid and viable,
    and identified some useful literature about the problem that you are trying to solve or investigate; One reason for doing this is to show that you will be able
    to find the useful and interesting ideas and information that your project will need. This requires references!
    It is also a good chance for you to practise the academic style of writing needed for your report.
}
Research into the existing solutions \cite{myfridgefood_myfridgefood_nodate} and \cite{supercook_supercook_nodate} found that both solutions provided a user account system
that would save a user's list of available ingredients. Both allow for recipes to be favourited/bookmarked which would then be shown on a separate page.
SuperCook additionally shows which of the user's favourite recipes can be made using available ingredients and which ingredients are missing.

However, neither of these apps has a system for sharing pantries between users or specifying disliked ingredients. These features would be useful for households with multiple people.
Additionally, the only way to update available ingredients is to manually add or remove them which is time-consuming and tedious. There is no tracking for the quantity of
each ingredient, so there is no way to know if the user has enough of an ingredient without checking manually.

Neither app tracks which recipes have been made recently, so there is no way to avoid suggesting similar recipes to ones that have been recently made which could lead to
users getting bored of eating the same meals over and over.

MyFridgeFood has a feature to submit new recipes which are shared between all users. This is a useful feature as it allows for the community to contribute to the database
of recipes, however, these cannot be kept private to just one user.

This research has led to the following requirements for the \chef{} app:
\begin{itemize}
    \item The system must allow for multiple users to share a pantry.
    \item The system must allow users to specify disliked ingredients that will not be used in suggested recipes.
    \item The system must allow users to submit new recipes, either to be shared with all users or kept private.
    \item The system should allow users to scan ingredients into their pantry.
    \item The system should track how much of each ingredient is available.
    \item The system should deduct from the available ingredients when a recipe is made.
    \item The system should track which recipes have been made recently and avoid suggesting similar recipes.
    \item The system should allow users to favourite recipes and have similar recipes suggested.
\end{itemize}
\todo{Should this research be in the main report?}
\todoinline{
    Indicate why the proposed project is of interest, both to you and more generally, and why it is useful and to whom.
    For example, where did the idea come from? Was it requested by your client, or did it arise from your supervisor’s research?
    Did some problem you encountered in real life or a paper that you read in the literature suggest a research question?
    Will it benefit a client – and how? What problems will you need to solve?  Will you be using interesting or unfamiliar technologies?
    What is challenging? (Remember that the project will stretch you beyond what you have achieved in other assignments.)
}
The project is of interest to me personally as I often struggle to decide what to eat and instead end up eating fast food,
the same meals over and over, or getting ready meals from the shop.

The idea for the project came due to the increase in cost of living, and the realization of how much food we bought that was
never consumed. I was searching for recipes which incorporated foods I had. This was time-consuming and laborious.
I had the idea to create an app where I could simply type in what food I had and would return how that could be incorporated
into a meal.

\todo{Background to Project}


\subsection{Proposed Work}
\todoinline{
    Be as specific as possible, someone with no experience should be able to understand what it is asking
}

The project will be the development of \chef{}, standing for Cooking Helper for Everyone's Fridge. I have chosen this name
as it is a catchy and memorable acronym that describes the purpose of the project, helping people to make use of
what they already have in their fridge.

\todoinline{
    •	A brief description of the work involved in carrying out the project and your approach to the development or investigation.
    Bring out the technical aspects as well as the general processes.
}
\chef{} will be a web application intended to be used on a mobile phone. It will allow users to create an account and log in,
then add the ingredients that they have either by manually entering them or by scanning the barcode on the packaging.
This ingredient will then be added to the user's virtual pantry and the user will be able to see what ingredients they have and how much of each.
If a barcode is not found in the database, the user can manually enter the type of ingredient and its quantity (weight, volume or quantity depending on the ingredient).
This information will then be stored in the database for future use \todo{All users or just the one inputting it?}.

Whole foods such as fruits and vegetables will be entered by the user manually, as they do not have barcodes. This will likely use an interface where
all supported ingredients are shown and the user clicks to add one to their fridge, clicking multiple times to add multiple of the same ingredient.
Another option would be to use a classifier to identify the type of fruit or vegetable from a picture, however, this may be less accurate and more
time-consuming than the manual method.

\chef{} will track what kinds of recipes each user has made recently and avoid suggesting similar recipes
to ones that the user has recently made, such as different pasta dishes. This will help to avoid users
getting bored of eating the same meals over and over.

Users within the same household will be able to invite each other to an in-app household that shares a fridge.
This will allow multiple people to add ingredients and use them to cook meals without having to sync up
what they have.

If a user does not like a certain ingredient, such as mushrooms, they can specify this in their profile and
the \chef{} app will not suggest recipes including it. This is necessary as other members of the household
may like the ingredient and therefore add it to the shared fridge.

When finding a recipe, users will be able to select which members of the household they are cooking for.
\chef{} will then find recipes that everyone likes in addition to using the available ingredients.

\todoinline{
    This section follows directly from the background and should give more details of what you are proposing to do.
    The project must exhibit a level of difficulty appropriate to final year honours BSc work, and be of a size that
    can be attempted in the time available; this section should define the topic and project work in enough detail
    for the markers to be sure that it is suitable. The more detail and discussion you produce at this stage, the stronger the foundation for the actual project work.
    You should emphasise the computing aspects you expect to be involved in, including those specifically relevant to your programme.
    Remember that you are undertaking the project as part of a BSc programme in a computing-related discipline, and avoid being side-tracked
    into areas that are not relevant to your course.)
    Make sure that you include the following:
    •	What areas or questions will you need to investigate to carry out the project work? What topic(s) will your literature review cover and how does it help the project?
    •	Any further details of the product to be built or practical investigative work that will help to define the size and scope of the project.
}
\todo{Proposed Work}

\todoinline{
    •	What computing technologies will be used and why?  (If you will be deciding this as part of the project work, say so, and indicate the possibilities.)
}
\chef{} will run on a local server with a Docker container for each component and a script to install dependencies and deploy the containers.
Using Docker will compartmentalize the project and reduce the work required to transfer to a hybrid cloud environment if required.
\todo{This is an implementation detail that may be better suited to the implementation section of the practical work}
\todoinline{
    This is one of the most important sections of your TOR. After reading this part of your TOR, the reader should know what would be involved in undertaking the project.
}
\todo{Proposed Work}

\subsection{Aims of Project}
\todo{Aims of Project}
\todoinline{
    The larger overarching goal of what the project is trying to investigate or do
    	An overall statement of what the project is intended to accomplish. This should be expressed in terms relating to the project,
    not personal achievements. There will normally be only one or two aims. If you have more than three, then revise them since they will normally be inappropriate as aims.
    The following are examples of project aims:
    •	“To investigate and analyse the importance of the role of schedulability in real-time systems.
    •	To develop a learning tool that would help people to understand the importance of schedulability and carry out a schedulability analysis of real-time systems.”
    •	“To compare the effects of flat and deep menu structures on users’ website navigation performance.”
    •	“To investigate how an e-card service is made and the types of database that can be used to store images.”
    •	“To build an e-card service to allow a user to manipulate images and create their e-card.”
    Note the English style in which aims are expressed and use the same style yourself.
}

\subsection{Objectives}
\todo{Objectives}
\todoinline{
    Specific and measurable goals that outline what will be accomplished. Derived from the aims and serve as building blocks of the project
    Should be:
      Specific, clear and precise
      Measurable for when it has been accomplished
      Achievable, realistic with constraints
      Relevant to the aims of the project
      Time-bound, give a deadline that is in line with the Gantt chart
    	Each objective should identify an expected outcome. By the end of the project, it should be obvious to you and your supervisor whether you have accomplished
    an objective or not, although it may be debatable as to how well you have accomplished it. Thus each objective must be clear, measurable and concrete.
    (The ‘SMART’ acronym is helpful here.) Phrase them so that it is clear what will show that the objective has been achieved. For example,
    	‘Create a requirements specification for…’  is more exact than,
    	‘Identify requirements for…’.
    	Projects for this module typically have 7-12 objectives, relating to areas such as:
    •	Literature review
    •	Establishing requirements
    •	Learning new/enhanced knowledge/skills,
    •	Creation of designs for a product
    •	Implementing a product
    •	Experimental work
    •	Production of items of systems/software documentation
    •	Testing a product
    •	Production of chapters of the project report
    •	Analysis of investigation results
    •	Evaluation of a product.
    •	Evaluation of the project process,
    •	Etc.
    If you have more than one aim, then you should be able to associate objective(s) with each aim. To produce your project plan, you will identify the tasks needed to achieve your objectives.
}

\subsection{Skills}
\todo{Skills}
\todoinline{
    The purpose of this section is to help you and your supervisor assess how strong a base of learning you are building on,
    and whether the enhancements of your knowledge and skills that the project will require are feasible. Projects should generally
    be relevant to your degree course, e.g. students taking Web Design and Development should do projects that use skills relevant
    to that area and build on modules from that course.
    Identify and list the skills that you will need to carry out the project work. You should explicitly identify both familiar areas
    of knowledge and skills and new/enhanced ones that the project will require. Against each, indicate which module that you have taken
    or are taking gives you those skills. If you will be acquiring skills during your project, say how this will be done.
}

\ifSubfilesClassLoaded{
    \printbibliography
}

\todoinline{
    7.	Sources of information/bibliography
    This is simply a reference list and bibliography, presented in the usual format. Include the sources you
    have consulted in preparing the Terms of Reference and any additional sources that you anticipate using
    during the work. It is not expected that you will yet have identified all the information that you will use.
    Ensure that you follow standard referencing guidelines.
}

\subsection{Resources - Statement of Hardware / Software Required}
\todo{Resources - Statement of Hardware / Software Required}

\chef is intended to use open-source software wherever possible and to be deployed on a local server which I already own.

A setup script will be created to install all dependencies and deploy the server. This will allow for easy deployment to a different server if required.
\todoinline{
    The purpose of this section is to ensure that all the resources needed for the project are available for it.
    Identify and justify the software, computer hardware or any other equipment that you require for the project.
    You should also indicate how it will be provided, and its purpose in the project. A bulleted list that gives this information for each item is sufficient.
    Consider carefully how you would cope if resources were not available; and what alternative resources might be employed. This is particularly
    important if you intend to use your equipment or that of an external body.
    Note that you will need to demonstrate your product at the university.
    It is important to plan how this will be achieved. Your list of resources should include those needed for the demonstration.
    Where relevant, you may use your computer for part or all of the practical work provided that it is suitable for the purpose,
    as agreed by the supervisor and that all proprietary software is properly licensed for the machine on which it is running.
    Similarly, hardware and software provided by a client may be used, under the same conditions.
    All software used in your project at any point must be used with the necessary licenses or permissions for that machine.
    This includes software on your machine and software used in the demonstration/viva. Do not make illegal copies,
    whatever the cost of the software in question. You may be able to find a free alternative, use software that is available in our labs,
    or choose a different approach. Software piracy is not acceptable under any circumstances.
}

\subsection{Assessment Criteria for Practical Computing Work}
\todo{Assessment Criteria for Practical Computing Work}
\todoinline{
    This is the part of the TOR that many students find the most difficult as you are asked to agree on the criteria on which your practical
    work will be assessed. This is very important for your final project marking.

    You should agree on these criteria with your supervisor and 2nd marker at the TOR review meeting, and you are strongly advised to review them
    with your supervisor early in Semester 2 to ensure that they still reflect the planned direction of your project. Changes may be agreed upon at
    that point, and need to be recorded in your eLogbook.

    (i)	Product
    If your project involves building a product, the product does not only consist of the final software or hardware but also of all the
    associated deliverables that you use to produce it, e.g. requirements specifications, design models, test plans and results, etc. These may vary widely between projects.
    The product should be marked in terms of two main criteria: ‘Fitness for Purpose’ and ‘Build Quality’.
    	Fitness for Purpose may be defined as “Have you built the right kind of product?”
    The main criterion for assessing this is whether the product meets its specified requirements as identified in the Research and Planning phase,
    and if not, why not? Other commonly usable criteria include:
    •	usability / HCI
    •	adequacy/completeness (bearing in mind that the product will typically be a prototype rather than a fully-fledged, complete product of practical use or commercial value),
    •	performance (e.g. good response time, sufficient data capacity),
    •	robustness
    •	error handling.
    Build Quality may be defined as “Have you built the product the right kind of way?”
    	A variety of criteria may be used in its assessment, for example:
    •	design quality,
    •	implementation quality,
    •	correspondence of implementation to the design,
    •	adherence to standards,
    •	quality of testing and validation,
    •	quality of specific documents.
    In some cases, it may be difficult to decide under which heading a criterion should fit: e.g. HCI might fit under either,
    depending on the focus of the project.  In these cases, decide on the most appropriate and ensure that the overall marking scheme fits in with the choice.
    List all the deliverables (finished software/hardware, requirement specification, design models, test plans, prototypes etc.)
    that will form part of your product. Then list the criteria on which fitness for purpose and build quality will be assessed.
    Some of these are provided as above, and you may add others.
    Please note that the mark percentage for practical work in the marking scheme is not only for the quality of the product.
    It also includes how you describe and justify the development process, the decisions made, and the techniques/tools used.


    (ii)	Quality of Practical Investigative Work

    If your project focuses on investigating a research question/hypothesis, in this section of the TOR, you must specify the criteria
    on which your investigative work will be assessed. These will vary widely between projects. You should list all the deliverables
    (e.g. tools built, survey instruments designed, equipment configurations set up; conduct of sessions with participants; use of tools and techniques, etc.)
    and choose criteria appropriate to the deliverables that you will produce. Some of these criteria may relate to the quality of deliverables produced;
    others may relate to the way that you have carried out your investigation. Observance of relevant ethical and safety guidelines must be included.
    Ensure that the deliverables and criteria chosen will demonstrate your use of practical computing skills relevant to your course.

    The specified deliverables and other evidence must demonstrate the effective use of practical computing skills relevant to the student’s programme;
    these are broadly defined and include e.g. carrying out usability evaluation, and use of appropriate tools. If insufficient evidence is provided,
    the student cannot score highly on this criterion.
    Remember that your markers will need to be able to assess your work on these criteria. You should indicate what evidence will be provided;
    This will usually be included in your appendices but may be elsewhere in your submission.
}

\subsection{Ethics, Social, Legal, and Professional Issues}
\todo{Ethics, Social, Legal, and Professional Issues}

\todoinline{
    You need to consider any ethical, social, legal and professional issues that might arise in your project. Please refer to the university's
    guidance on Ethics and Integrity for consultation. You need to discuss with your supervisor in which category your project will be classified
    in terms of ethics. If your project is considered a low or medium-risk project, you need to justify the reasons in this session
    and provide action plans on human participants, data collection, data management recruitment, risk assessment, health and safety, and any other relevant issues.

    Ethics approval is a fail/pass component. You need to fill out an Ethics Approval Form and submit it together with the TOR document
    to provide enough information for review. Failure to submit both documents promptly will result in you not being able to start your project,
    and consequently a delay in the project completion. Failure to pass the ethics approval will result in failing the module assessment.
    If a risk assessment is required, you should also fill out a Risk Assessment Form and attach it to the TOR document.
}

\subsection{Project Plan - Schedule of Activities}
\todo{Project Plan - Schedule of Activities, use a Gantt chart broken down by week}
\todoinline{
    A detailed statement of the stages of the project is required. A Gantt chart should be used for expressing the schedule, and
    together with supporting information. Identify the tasks needed to achieve each objective, estimate how long they will take,
    plan when they should be done, and include these tasks in your Gantt chart.
    Your schedule must indicate for each task the number of hours work required to accomplish it and the elapsed
    time over which this work should be done. The schedule should be expressed in weeks. Months do not allow for sufficiently detailed planning and control.
    Days are too detailed; in practice, you will not yet know your daily workload in the second semester, so you cannot meaningfully plan down to the day.
    It is sensible to consult your assignment schedule when you make your project plan and to avoid overloading yourself at times when you have other major
    commitments, but remember that your project is the largest and most important module of the year!
}

\ifSubfilesClassLoaded{
    % TODO: Remove once done
    \todos
}{}

\end{document}
