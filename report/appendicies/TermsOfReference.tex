\documentclass[../CHEFCookingHelperForEveryonesFridge.tex]{subfiles}
% TODO: Remove once done
\ifSubfilesClassLoaded{
    \usepackage{draftwatermark}
}

\renewcommand{\thesubsection}{\arabic{subsection}.}

\begin{document}

\begin{table}[h!]
    \centering
    \begin{tabular}{|c|c|}
        \hline
        \multicolumn{2}{|c|}{KV6003: Individual Computing Project Terms of References} \\\hline
        Student name: & Kieran Knowles \\\hline
        Student ID: & w20013000 \\\hline
        Course: & BSC (HONS) Computer Science \\\hline
        Project Title: & \chef{} - Cooking Helper for Everyone's Fridge \\\hline
        Supervisor: & Nick Dalton \\\hline
        2\textsuperscript{nd} Marker: & Marta Cecchinato \\\hline
    \end{tabular}
\end{table}

\subsection{Background to Project}
As of 2020, UK households collectively wasted 4.5 million tons of food every year, or nearly 20\% of all purchased food.
Whilst aware that food waste is increasing among consumers, \textquote*{many do not yet acknowledge that this is an issue relevant to them or are not
yet concerned enough to act} \cite{wrap_uk_2020}.

With the current cost of living crisis, where inflation, especially in food costs, is vastly outpacing wage growth \cite{office_for_national_statistics_average_2023}
\cite{francis-devine_rising_2023}, families are looking to cut costs wherever possible. One way to do this is to reduce food waste which is a
primary aim of this project.

One possible reason for this is consumers not knowing what to do with the food they have in their fridge, and therefore not using it before it goes off.
Manually looking through a recipe book and checking that you have all the ingredients is a time-consuming process, and therefore people are less likely to do it.

This project aims to create a web application that tracks what ingredients the user has and then suggests recipes that can be made with those ingredients.
This will help to reduce food waste by helping users work with what they already have rather than buying more food and letting what they have go off.

The project is of interest to me personally as I often struggle to decide what to eat and instead end up eating fast food,
the same meals over and over, or getting ready meals from the shop.

The idea for the project came due to the increase in cost of living, and the realization of how much food we bought that was
never consumed. I was searching for recipes that incorporated foods I had. This was time-consuming and laborious.
I had the idea to create an app where I could simply type in what food I had and would return how that could be incorporated
into a meal.

\subsection{Proposed Work}
The project will be the development of \chef{}, standing for Cooking Helper for Everyone's Fridge. I have chosen this name
as it is a catchy and memorable acronym that describes the purpose of the project, helping people to make use of
what they already have in their fridge.

\chef{} will be a web application intended to be used on a mobile phone. It will allow users to create an account and log in,
then add the ingredients that they have either by manually entering them or by scanning the barcode on the packaging.
This ingredient will then be added to the user's virtual fridge and the user will be able to see what ingredients they have and how much of each.
If a barcode is not found in the database, the user can manually enter the type of ingredient and its quantity (weight, volume, or quantity depending on the ingredient).
This information will then be stored in the database for future use. As \chef{} will likely use an open-source dataset for barcodes, it could provide an option
for users to submit their new barcodes to the dataset to help improve it.

Whole foods such as fruits and vegetables will be entered by the user manually, as they do not have barcodes. This will likely use an interface where
all supported ingredients are shown and the user clicks to add one to their fridge, clicking multiple times to add multiple of the same ingredient.
Another option would be to use a classifier to identify the type of fruit or vegetable from a picture, however, this may be less accurate and more
time-consuming than the manual method.

\chef{} will track what kinds of recipes each user has made recently and avoid suggesting similar recipes
to ones that the user has recently made, such as different pasta dishes. This will help to avoid users
getting bored of eating the same meals over and over.

Users within the same household will be able to invite each other to an in-app household that shares a fridge.
This will allow multiple people to add ingredients and use them to cook meals without having to sync up
what they have.

If a user does not like a certain ingredient, such as mushrooms, they can specify this in their profile and
the \chef{} app will not suggest recipes including it. This is necessary as other members of the household
may like the ingredient and therefore add it to the shared fridge.

When finding a recipe, users will be able to select which members of the household they are cooking for.
\chef{} will then find recipes that everyone likes in addition to using the available ingredients.

A competitor analysis will be performed to identify features that are common among similar apps and to identify potential
improvements that could be made for \chef{}. This has been started in the \mainfilehyperref{sec:competitor_analysis}{Competitor Analysis}
section of the main report with the apps  \cite{myfridgefood_myfridgefood_nodate} and \cite{supercook_supercook_nodate}.
This has identified the following requirements for the \chef{} app which will be expanded upon in the requirements section
of the main report.

\subsubsection{Must Have}
\begin{itemize}
    \item The system must allow for multiple users to share a virtual fridge.
    \item The system must allow for dietary requirements (e.g., allergies) to be specified for each user and must not suggest recipes that do not meet these requirements.
    \item The system must allow users to specify disliked ingredients that will not be used in suggested recipes.
    \item The system must allow users to submit new recipes, either to be shared with all users or kept private.
\end{itemize}

\subsubsection{Should Have}
\begin{itemize}
    \item The system should allow users to scan ingredients into their virtual fridge.
    \item The system should track how much of each ingredient is available.
    \item The system should deduct from the available ingredients when a recipe is made.
    \item The system should track which recipes have been made recently and avoid suggesting similar recipes.
    \item The system should allow users to favorite recipes and have similar recipes suggested.
    \item All ingredient lists should use metric units by default.
\end{itemize}

\subsubsection{Could Have}
\begin{itemize}
    \item The system could allow for the scanning of a receipt to automatically add ingredients to the virtual fridge.
    \item The system could provide an estimate of the cost of each recipe.
    \item The system could provide an estimate of how much waste/money has been saved by using it.
    \item The system could allow users to contribute to the datasets accessed by the app.
\end{itemize}

The technologies required for the \chef{} app are discussed below in the \hyperref[sec:resources_statement]{Resources} section.

\subsection{Aims of Project}

\begin{enumerate}
    \item To create a system to reduce food waste by helping users make use of ingredients they already have.
    \item To encourage users to try new recipes that they may not have known they could make.
\end{enumerate}

\subsection{Objectives}

\begin{enumerate}
    \item Research existing solutions to identify what features are useful and what can be improved upon.
    \item Research potential data sources for ingredients and recipes.
    \item Research potential technologies for identifying ingredients from barcodes, images, and receipts.
    \item Expand on the existing requirements to create a full set of requirements for the \chef{} app.
    \item Develop the \chef{} application in line with the requirements.
    \item Test the \chef{} application to ensure that it meets the requirements.
    \item Evaluate the final product to determine if it meets the aims of the project.
\end{enumerate}

\subsection{Skills}
\todo{Skills}
\todoinline{
    The purpose of this section is to help you and your supervisor assess how strong a base of learning you are building on,
    and whether the enhancements of your knowledge and skills that the project will require are feasible. Projects should generally
    be relevant to your degree course, e.g. students taking Web Design and Development should do projects that use skills relevant
    to that area and build on modules from that course.
    Identify and list the skills that you will need to carry out the project work. You should explicitly identify both familiar areas
    of knowledge and skills and new/enhanced ones that the project will require. Against each, indicate which module that you have taken
    or are taking gives you those skills. If you will be acquiring skills during your project, say how this will be done.
}

\ifSubfilesClassLoaded{
    % Print the bibliography on the same page
    % From bib latex docs
    \ifSubfilesClassLoaded{
        \defbibheading{bibliography}[\bibname]{
            \subsection{#1}
            \markboth{#1}{#1}
        }
    }{}
    \printbibliography
}{
    \subsection{References}
    See the \hyperref[sec:reference_list]{References} section of the main report for the main reference list, including those that were consulted when preparing the TOR.
}

\subsection{Resources - Statement of Hardware / Software Required}
\label{sec:resources_statement}

\chef{} is intended to use open-source software wherever possible and to be deployed on a local server which I already own.

A setup script will be created to install all dependencies and deploy the server. This will allow for quick and easy deployment to a new server
if required and for a reproducible development environment.

Recipes and barcodes will be sourced from an external API, preferably open-source. This will remove the need to manually enter data and allow for
\chef{} to automatically update with new recipes and barcodes as its data sources are updated.

If needed, more demanding tasks such as image recognition will be offloaded to a cloud service such as Azure as the server I own has limited hardware.

Frontend code will be written in TypeScript using the React framework while the backend is currently undecided.

\subsection{Assessment Criteria for Practical Computing Work}
\todo{Assessment Criteria for Practical Computing Work}
\todoinline{
    This is the part of the TOR that many students find the most difficult as you are asked to agree on the criteria on which your practical
    work will be assessed. This is very important for your final project marking.

    You should agree on these criteria with your supervisor and 2nd marker at the TOR review meeting, and you are strongly advised to review them
    with your supervisor early in Semester 2 to ensure that they still reflect the planned direction of your project. Changes may be agreed upon at
    that point, and need to be recorded in your eLogbook.

    (i)	Product
    If your project involves building a product, the product does not only consist of the final software or hardware but also of all the
    associated deliverables that you use to produce it, e.g. requirements specifications, design models, test plans and results, etc. These may vary widely between projects.
    The product should be marked in terms of two main criteria: ‘Fitness for Purpose’ and ‘Build Quality’.
    	Fitness for Purpose may be defined as “Have you built the right kind of product?”
    The main criterion for assessing this is whether the product meets its specified requirements as identified in the Research and Planning phase,
    and if not, why not? Other commonly usable criteria include:
    •	usability / HCI
    •	adequacy/completeness (bearing in mind that the product will typically be a prototype rather than a fully-fledged, complete product of practical use or commercial value),
    •	performance (e.g. good response time, sufficient data capacity),
    •	robustness
    •	error handling.
    Build Quality may be defined as “Have you built the product the right kind of way?”
    	A variety of criteria may be used in its assessment, for example:
    •	design quality,
    •	implementation quality,
    •	correspondence of implementation to the design,
    •	adherence to standards,
    •	quality of testing and validation,
    •	quality of specific documents.
    In some cases, it may be difficult to decide under which heading a criterion should fit: e.g. HCI might fit under either,
    depending on the focus of the project.  In these cases, decide on the most appropriate and ensure that the overall marking scheme fits in with the choice.
    List all the deliverables (finished software/hardware, requirement specification, design models, test plans, prototypes, etc.)
    that will form part of your product. Then list the criteria on which fitness for purpose and build quality will be assessed.
    Some of these are provided as above, and you may add others.
    Please note that the mark percentage for practical work in the marking scheme is not only for the quality of the product.
    It also includes how you describe and justify the development process, the decisions made, and the techniques/tools used.


    (ii)	Quality of Practical Investigative Work

    If your project focuses on investigating a research question/hypothesis, in this section of the TOR, you must specify the criteria
    on which your investigative work will be assessed. These will vary widely between projects. You should list all the deliverables
    (e.g. tools built, survey instruments designed, equipment configurations set up; conduct of sessions with participants; use of tools and techniques, etc.)
    and choose criteria appropriate to the deliverables that you will produce. Some of these criteria may relate to the quality of deliverables produced;
    others may relate to the way that you have carried out your investigation. Observance of relevant ethical and safety guidelines must be included.
    Ensure that the deliverables and criteria chosen will demonstrate your use of practical computing skills relevant to your course.

    The specified deliverables and other evidence must demonstrate the effective use of practical computing skills relevant to the student's program;
    these are broadly defined and include e.g. carrying out usability evaluation, and use of appropriate tools. If insufficient evidence is provided,
    the student cannot score highly on this criterion.
    Remember that your markers will need to be able to assess your work on these criteria. You should indicate what evidence will be provided;
    This will usually be included in your appendices but may be elsewhere in your submission.
}

\subsection{Ethics, Social, Legal, and Professional Issues}

\ifSubfilesClassLoaded{
    {
        \renewcommand{\subsection}[1]{\subsubsection{#1}}
        \subfile{EthicsApprovalForm.tex}
    }
}{
    See the \hyperref[sec:ethics_approval]{Ethics Approval Form} appendix for the ethics approval form.
}

\subsection{Project Plan - Schedule of Activities}

\begin{ganttchart}[hgrid, vgrid, x unit=1cm]{6}{12}
    \gantttitle{2023 Week 6 to Christmas Break}{7} \\
    \gantttitlelist{6,...,12}{1} \\
    \ganttbar[name=research_solutions]{Research existing solutions}{6}{6} \\
    \ganttbar[name=research_sources]{Research data sources}{7}{7} \\
    \ganttbar[name=research_technologies]{Research data input technologies}{8}{8} \\

    \ganttbar[name=requirements]{Expand on requirements}{9}{12}
    \ganttlink[link type=dr]{research_solutions}{requirements}
    \ganttlink[link type=dr]{research_sources}{requirements}
    \ganttlink[link type=dr]{research_technologies}{requirements}
\end{ganttchart}

\begin{ganttchart}[hgrid, vgrid]{1}{12}
    \gantttitle{2024 Week 1 to Submission}{12} \\
    \gantttitlelist{1,...,12}{1} \\
\end{ganttchart}

\todo{Project Plan - Schedule of Activities, use a Gantt chart broken down by week}
\todoinline{
    A detailed statement of the stages of the project is required. A Gantt chart should be used for expressing the schedule, and
    together with supporting information. Identify the tasks needed to achieve each objective, estimate how long they will take,
    plan when they should be done, and include these tasks in your Gantt chart.
    Your schedule must indicate for each task the number of hours work required to accomplish it and the elapsed
    time over which this work should be done. The schedule should be expressed in weeks. Months do not allow for sufficiently detailed planning and control.
    Days are too detailed; in practice, you will not yet know your daily workload in the second semester, so you cannot meaningfully plan down to the day.
    It is sensible to consult your assignment schedule when you make your project plan and to avoid overloading yourself at times when you have other major
    commitments, but remember that your project is the largest and most important module of the year!
}

\ifSubfilesClassLoaded{
    % TODO: Remove once done
    % Using a list from the ethics form
    % \todos
}{}

\end{document}
