\documentclass[../CHEFCookingHelperForEveryonesFridge.tex]{subfiles}

\begin{document}
\todo{Format and add TODOs}
Student Project Ethics Approval Form
You should use this document if your project is not high risk. Please complete this document and discuss your study with your supervisor before you collect any data. Failure to complete this document and have all aspects signed off and approved by your supervisor and 2nd marker results in a fail in your final module result and may risk a case of Academic Misconduct.
Please ensure that your project meets the conditions of the existing module level ethics application (available on Blackboard). If it does not, then you will need to submit a full ethics application instead via the main university ethics system.

\begin{table}[h!]
    \centering
    \begin{tabular}{|c|c|}
        \hline
        Student Name: & Kieran Knowles \\\hline
        Student ID: & w20013000 \\\hline
        Programme Name: & BSc Computer Science \\\hline
        Project Title: & \chef \\\hline
        Supervisor Name: & \todo{Supervisor name} \\\hline
        Second Marker: & \todo{Second marker} \\\hline
        What type of study are you using (check all that apply): &
            \todo{Fill this out}
            \todo{Fix invalid char for checkboxes}
            \todo{How do i do newline within a table?}
            ☐   Questionnaire or Survey
            ☐   User Studies
            ☐   Data Generated by Systems
            ☐   Secondary Data Analysis
            ☐   No data collected from humans \\\hline
    \end{tabular}
\end{table}


Please answer the following questions and complete all information in full:
1.	Human Participants: does your study involve human participants                      YES/NO
If YES, please answer the following questions and ensure that you include your participant information sheet, participant consent sheet and any participant recruitment materials/permission letters for participants in Appendix B:
1a) Who are your participants and what is the inclusion criteria you will be using?

1b) How many participants will you recruit and from where?

1c) Are there any exclusion criteria (reasons why people should not participate)?


2.	Data Collection: Will your study collect any primary data or use any secondary data not in the public domain?                              						YES/NO

Please complete the following questions, noting that somebody should be able to read this and replicate your approach:
2a) What type of data are you going to use? (Identify main types of information/data)
2b) What procedures will you use to collect data (include all equipment/methods you plan to use)
2c) What methods will you use to analyse this data?

3.	Data Management
Standard phrases have been added to the information sheet (available on Blackboard). In rare instances, these may not be appropriate for your study. If not, please describe any additional data management procedures below:

4.	Risk Assessment, Health and Safety
All research activity carried out by Northumbria University is subject to risk assessment and health and safety issues. Depending on the nature of your research work, you may need to use one of the risk assessments below and/or complete a Project Risk Assessment in discussion with your supervisor. Once you have identified risks and associated health and safety issues, you may need to consult relevant technical and other staff for further advice and guidance. Further information including a blank risk assessment form for research can be found here: Risk Assessment (northumbria.ac.uk).
Please check this box after you have read and understood ethics and health and safety information
\todo{Fix invalid char for checkboxes}
☐   I confirm I have read the University's health and safety policy and ethics policy. I have read and understood the requirement for the mandatory completion of risk assessments and that my study does not deviate from the module level approval ethics information on Blackboard:  Relevant risk assessments are listed in the ethics application. If your project needs additional risk assessments, then you will need to submit a new ethics application. Please identify the elements of the listed risk assessment that are relevant for your study and the risk assessment(s) you are working with. Note that these are only relevant if you are collecting data face-to-face.
Please check the relevant boxes:
☐   No physical risks
☐   HL\textunderscore RISK\textunderscore 173 Testing in an external environment
☐   HL\textunderscore RISK\textunderscore 722 Face-to-face interview
☐   HL\textunderscore RISK\textunderscore 727 Group interview

Supervisor (and/or Second Marker where appropriate) to assess using the following criteria:

Tutor sign off
Ethics form complete	☐

Ethical concerns acknowledged	☐

Research tool(s) checked	☐

All relevant forms included (consent etc.)	☐

Is not high risk	☐

  
Appendix A: Terms of Reference
You MUST include your Terms of Reference document to provide information on project aims, objective, research methodology, resources, and ethics, social, legal and professional considerations.


Appendix B: Participant Information \& Consent Form
Note: this section MUST be completed if you are including human participants in your study

Please include here your participant information sheet, participant consent form plus any participant recruitment materials and permission letters.

Appendix C: Risk Assessment Form
If your project involves any healthy, safety risks, please also include a risk assessment form.

% TODO: Remove once done
\ifSubfilesClassLoaded{
    \todos
}{}

\end{document}
