\documentclass[../CHEFCookingHelperForEveryonesFridge.tex]{subfiles}

\begin{document}

\begin{table}[h!]
    \centering
    \begin{tabular}{|c|c|}
        \hline
        \multicolumn{2}{|c|}{KV6003: Individual Computing Project Student eLogbook} \\\hline
        Student name and ID: & Kieran Knowles w20013000 \\\hline
        Academic year: & 2023-2024 \\\hline
        Programme: & Pure Computer Science \\\hline
        Project title: & \chef{} \\\hline
        Supervisor: & Nick Dalton \\\hline
        Supervisor email and room number: & \makecell[c]{
            CIS 304 \\
            \href{mailto:nick.dalton@northumbria.ac.uk}{nick.dalton@northumbria.ac.uk}
        } \\\hline
        Second marker: & Marta Cecchinato \\\hline
        Second marker email and room number: & \makecell[c]{
            CIS 307 \\
            \href{mailto:marta.cecchinato@northumbria.ac.uk}{marta.cecchinato@northumbria.ac.uk}
        } \\\hline
    \end{tabular}
\end{table}

\textbf{\underline{PROJECT E LOGBOOK INSTRUCTIONS}}

This eLogbook can be used to record the progress of work on the project. It covers all the stages of the
work and it is strongly recommended that it be \ul{completed in full by the student.}

\textit{
    Note: if this eLogbook is not used by the student, \ul{a suitable alternative should be adopted in agreement with
    the Project Supervisor;} the alternative should be accessible by the project tutor if the need arises and able
    to be included as electronic appendices in the final dissertation submission (e.g. consider how the alternative
    can be exported).
}

This eLogbook or other form of project log should be submitted upon completion of the project.
You should also use excerpts from it to evidence claims made in the evaluation section of your report.

\textbf{BEFORE EACH MEETING}

Before each meeting, \textbf{\ul{YOU} must complete the following sections of the relevant weekly log form
(\colorbox{\logbookshadecolour}{shaded in orange}) and \ul{SHARE} the revised complete eLogbook with your Supervisor
(e.g. via email, shared file space):}

\begin{itemize}
    \item Date and time of the meeting;
    \item Brief description of work done since the last meeting;
    \item Number of hours spent on the project since the last meeting;
    \item Questions/items to discuss at the meeting (agenda).
\end{itemize}

\textbf{AFTER EACH MEETING}

\textbf{By the end of the meeting, YOU and your SUPERVISOR must complete the other parts of the form for the week
(not shaded).} After the remaining sections are completed during the meeting, the revised eLogbook should be once
again saved and shared (i.e. via email or shared file space).

If, for any reason, you have more than one meeting in a week or hold a meeting outside of teaching weeks,
simply copy and paste a blank table to create a new entry, taking care to place it in chronological order and
change the heading to something meaningful. For example, if you have two meetings in one week you could use
Week 3a and Week 3b or Week 3.1 and Week 3.2 to indicate the occurrences; alternatively, if you meet outside
allocated teaching weeks for any reason, you could add the meeting date to the heading.

It is recommended that you use the \enquote*{Navigation Pane} in Word to provide easy and quick access to each respective
log form and/or other component.  On the top menu, simply select \enquote*{View} and tick the \enquote*{Navigation Pane} box, which
is located in the \enquote*{Show} section towards the left-hand side.

\logbookentry{
    Semester 1 Week 5a
}{
    Thursday 2nd November 16:00-17:00
}{
    Begin working on the TOR and ethics approval and start the competitor analysis
    in the main report
}{
    N/A
}{
    What needs to be said in the Data Collection and Data Management sections of
    the ethics approval? Can I use customer reviews as a data source?
}{
    Finalize the TOR and ethics approval.
}{
    TOR and ethics approval
}{
    Friday 3rd November 11:05-11:35
}{
    Yes
}

\logbookentry{
    Semester 1 Week 5b
}{
    Friday 4th November 11:05-11:35
}{
    Finalize the ethics approval and complete most of the TOR.
}{
    2 hours.
}{
    Review the TOR and submit it for review. Clarify which referencing style to use.
    Begin work on implementing the project.
}{
    Finish and submit the TOR. Add machine learning to group similar recipes and suggest them.
}{
    The TOR needs to do something different to existing products such as incorporating machine learning.
}{
    Thursday 9th November 16:00-17:00
}{
    Yes
}

\logbookentry{
    Semester 1 Week 6
}{
    Thursday 9th November 16:00-17:00
}{
    Start implementing the app by getting data imported.
}{
    5 hours.
}{
    N/A
}{
    Complete data import and make a start on tracking ingredients.
}{
    App backend.
}{
    Thursday 16th November 16:00-17:00
}{
    Yes
}

\logbookentry{
    Semester 1 Week 7
}{
    Thursday 16th November 16:00-17:00
}{
    Get the data import to a state I'm happy with.
}{
    15 hours.
}{
    N/A
}{
    Get ingredient tracking and recipe suggestions implemented.
}{
    Data import component of the app.
}{
    Thursday 30th November 16:00-17:00
}{
    Yes
}

\logbookentry{
    Semester 1 Week 9
}{
    Thursday 30th November 16:00-17:00
}{
    Implement basic recipe suggestion
}{
    10 hours.
}{
    How should sources for programming be cited?
}{
    Design and make a start on implementing the front end. Look into unit testing and machine learning for recipe suggestions using a single-layer perceptron model.
}{
    N/A
}{
    Thursday 7th December
}{
    Yes
}

\logbookentry{
    Semester 1 Week 10
}{
    Thursday 7th November 16:00-17:00
}{
    Start implementing the front end. The user can view their ingredients, add ingredients manually, and view available recipes.
}{
    20 hours.
}{
    Can copilot be used for generating code?
}{
    Continue working on the app. Look further into unit testing.
}{
    Frontend app. Copilot can be used for the code.
}{
    Thursday 14th December
}{
    Yes
}

\logbookentry{
    Semester 1 Week 11
}{
    Thursday 14th November 16:00-17:00
}{
    Show the ingredients required for a recipe. Allow filtering to allow one or more missing ingredients. Check ingredient amounts.
}{
    TODO: Number of hours since the last meeting
}{
    TODO: Questions/items to discuss at the meeting
}{
    TODO: Agreed tasks for the next meeting
}{
    TODO: Documents discussed/other issues
}{
    TODO: Date/time of next meeting
}{
    TODO: As scheduled?
}

\logbookentry{
    Semester 1 Week 12
}{
    Thursday 21st November 16:00-17:00
}{
    Implement machine learning into the app to find similar recipes. Start documenting ML implementation.
}{
    15
}{
    N/A
}{
    Complete the app and start testing.
}{
    App
}{
    TODO: Date/time of next meeting
}{
    Yes
}

\end{document}
