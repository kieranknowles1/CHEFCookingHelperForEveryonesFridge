\todoinline{
    Introduction
    The introduction provides a fuller overview of the work to be done than is given in the abstract and sets the scene for the detail provided in subsequent chapters.
    You may wish to draw on the background section written for your Terms of Reference, but the introduction should not simply reproduce parts of the TOR: you will develop
    a fuller understanding during the course of the project. The introduction is normally written (or at least finalised) at the end of the project, and is written
    retrospectively, i.e. it says what you did, not what you are going to do. This section is quite straightforward, but ensure that you have all the elements listed in
    the marking scheme. It is quite common for students to lose a mark or two by missing something out.
}

\section{The Need to Reduce Food Waste}
Gloally, nearly one third of food produced is wasted with figures being highest
in industrialized countries, leading to unnecessary greenhouse gas emissions and
waste of natural resources.~\cite{gustavsson_global_2011} A signigicant proportion
of this wastage occurs at the household level, totalling approximatly 50\% of all
food waste as of 2012.~\cite{stenmarck_estimates_2016} Overall, food waste contributes
to a significant amount of the global warming caused by the entire food supply
chain.~\cite{scherhaufer_environmental_2018}

\section{Algorithims and Datasets}
The intention of the project is to create a system based on known algorithms and
datasets that can be used to reduce food waste. Because these algorithms and
datasets have been proven to work for recipe suggestion, the project will focus
instead on the implementation of the system and the evaluation of its effectiveness.
