% In an implementation chapter, you are describing and justifying how you implemented your product, which for a software product means at the code level.
% Do not attempt to describe every detail. For a software product, do not include large sections of program code. Any code presented should be to illustrate important
% and interesting features. You might want to describe the data structures you elected to use, e.g. in Java a LinkedList or a Vector, and explain why you chose the one
% you did. If there were any interesting low-level algorithms, you should describe these. If you feel it is important to put a significant volume of program code into
% your report, put it in an appendix and reference the appendix. (The appendix usually contains examples of your code, but the place for your full code is in your OneDrive
% product folder.)

% Writing about program code can sometimes cause a student problems. You should be able to find good examples of articles that discuss implementations on the web.
% Read them and learn from how they do it. For Java a good source of examples can be found at JavaWorld.com.
% Pick out the key parts of your implementation and provide a rationale for them. During your attempts to implement your product, you may have had to face unforeseen
% problems. Explain how you overcame them. They may have caused you to modify the original design. Discuss the implications of those changes.
