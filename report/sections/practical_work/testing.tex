% Testing is not part of evaluation. It is the last part of your development activities. It is about how you checked to make sure your product was a viable and robust
% piece of software.

% A testing chapter of your report indicates the approach you have taken to verifying and validating your system. You should not merely list the tests performed. Rather,
% you should discuss how tests were selected, why they are sufficient, why a reader should believe that no important tests were omitted, and why the reader should believe
% that the system will really operate as desired.
% You should explain your overall strategy for testing: black box and/or white box, top down and/or bottom up, kinds of test beds or test drivers used, sources of test data,
% test suites, coverage metrics, compile-time checks vs. run-time assertions, reasoning about your code, etc. You might want to use different techniques (or combinations
% of techniques) in different parts of the program. In each case, justify your decisions. It is not necessary to describe the techniques; the reader knows about them.
% Tell the reader what you used and why in the context of your product. If you carried out usability testing, explain your approach to this.

% Explain what classes of errors you expect to find (and not to find!) with your strategy. Discuss what aspects of the design make it hard or easy to validate.
% Summarise the testing that has been accomplished and what if any remains. Which modules have been tested, and how thoroughly? Indicate the degree of confidence in the code:
% what kinds of fault have been eliminated? What kinds might remain?  Do not include large volumes of tables purporting to be a test log here. These should be in the product
% documentation.

% 	Projects focusing on the investigative work
% You should discuss the design, investigation, and results of your investigative work.
% Design
% You should discuss the detailed design of your research or investigation. You need to distinguish between what you planned to do and what happened when you actually
% carried out the work. Think about all the decisions that you had to make as you planned the work, and explain why you chose the approaches that you took and rejected others.
% For example, if you planned to carry out a study of how people use menu structures on web pages, you probably made decisions such as: What software will I use? How many
% people will be involved? How will they be chosen? What alternative tasks will I give them? On what basis will I divide the people into groups and / or assign them
% to different tasks? What exactly will I measure, and how, and what equipment will I need? How will I analyse the resulting data? If you are comparing the performance
% of networks using different configurations, how will you set them up, what tests will you carry out, how will you measure the performance, what data will you use, how
% will you analyse it, etc? How will you follow any relevant ethical or safety guidelines? You should justify these decisions, showing that you considered alternative
% solutions carefully.
