% The final part of is the presentation of your results. You may have quantitative or qualitative data, or even both. The best way of presenting your results will be determined
% by the type of data and the nature of your investigation but will usually involve summarising the data in some way.  Data analysis is too large and varied a topic to discuss
% in detail here, and how you do it will be very dependent on the project that you are doing. You are likely to have found out about appropriate methods as part of your
% project. In some cases, it may be appropriate to present calculations, e.g. to demonstrate how performance figures are derived. If you are doing statistical analyses,
% it will be helpful include levels of significance with the results where applicable. The presentation of qualitative data may involve summarising it, identifying significant
% factors, including representative examples, discussing interesting cases or critical incidents in depth, the use of quotations and illustrations, or identifying significant
% categories of content from textual data and looking at how often they occur.  For example, if you had been interviewing people about their use of information, you might
% identify categories related to the type of information, the method of access, etc.
% It may be helpful to use diagrams, charts or tables to present the work or the results. These should come with enough explanation for the reader to make sense of what
% you are showing.

% In general, raw data in the body of your report will be limited to small elements or examples that that you are discussing. Further data can go in your appendices or
% (if bulky) on your OneDrive folder, and you should tell the reader where to look for it. Identifiable personal data should never be included in any part of your submission.
% If you need to refer to individuals, you can mention ‘Participant A’ in your study, or in organisational settings it may be appropriate to refer to someone by their job title.
