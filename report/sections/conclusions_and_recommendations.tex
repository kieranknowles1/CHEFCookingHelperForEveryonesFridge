\todoinline{
    Conclusions and Recommendations
    An effective set of conclusions should not introduce new material. Instead, it should briefly draw out, summarize, combine and reiterate
    the main points that have been made in the body of the project report and present opinions based on them. The conclusions section should include a
    summary of what has been achieved, and of the main results of the project. To some extent this will involve summarizing aspects of the evaluation.

    The main conclusions will relate to the problem that you tackled: what answers and solutions did you find to your research question; what did you conclude
    about your hypotheses; what product did you build?  You should also sum up what you achieved, and how far your aims and objectives were met. You may also
    have drawn other conclusions from the work, perhaps about the effectiveness of methods or tools used.

    It is quite likely that by the end of your project you will not have achieved all that you planned at the start; and in any case, your ideas will have grown
    during the course of the project beyond what you could hope to do within the available time. The recommendations will focus on further work: this is where
    you describe your unrealized ideas. It is where you tell the reader what extra you wish you could have done to benefit this subject area. Try to look
    beyond the work that yourself have done to the subject context. A good set of recommendations can provide the basis for a future project. You may also
    have more general recommendations for other people working in this field.

    1-3 pages. Can be bullet points. What would be the next steps in taking this forward.
    Don't introduce new material here. Just summarise and bring everything together.
}
