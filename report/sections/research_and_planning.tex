\todoinline{
    Research and Planning
    The research and planning section may consist of several chapters. The exact number will depend upon the nature of the work you are undertaking.
    Typically, this part of the report should provide the reader with information they will need to know in order to appreciate and understand the work you have done in
    the rest of the project. You should assume a computer literate reader but not a specialist in your particular topic.
    The research and planning follow on from the background given in the introduction. The marking scheme asks for the following five areas to be covered:
    •	Clear identification and analysis of the problem to be looked at, identifying the key technical or other problems to be solved. Necessary background material
    that goes beyond the scope of the introduction may be included.
    •	A critical review of literature related to the topic; this will normally address some combination of the underlying principles of the problem area and
    possible approaches to solving the problem. The relevance of these ideas to the project should be clear.
    •	Discussion of approaches appropriate for the solution of the problem.
    •	A discussion and justification of the product requirements if a product will be built.
    •	Explanation and justification of the tools and techniques to be used in the project work.

    Note that while all these elements are suggested, they do not necessarily map on to five separate chapters of your report. For example, a discussion of approaches to
    solving the problem may fit naturally at the conclusion of your literature review. “Tools and techniques” here refer to what you use to build the product or conduct
    the investigation, while “approaches” is used to include higher-level issues such as an overall strategy or architecture, the choice of implementing one algorithm rather
    than another to carry out a task, a general type of solution, etc. You should arrange the material in chapters in whatever way that best suits your project. Some projects
    will not even require all five areas. Ensure that you include everything that is necessary for your project.

    Your research and planning should include a discussion of the wider issues, and critically examine the methods that might be used in solving the problem and any constraints
    that apply. Beware of presenting a shallow treatment of the subject which might be obtained from standard texts. You are expected to support your argument by exploring
    academic literature which is seminal and up to date.

    The background and problem description should describe the actual problem you are looking at and set it in context. You should tell the reader about the computing problem(s)
    to be investigated, the product to be built, etc. as relevant, building on the brief overview given in the introduction. Bring out features of the problem you would not
    expect the reader to know about. Do not state the obvious.

    The literature review will be a very important component of the research. Your literature review may discuss the problem that you are trying to solve, the question that
    you are trying to answer, possible solutions, methods of investigation, etc. as appropriate to your needs. You need to find out the state of the art - what others have
    done in this area, what solutions have been proposed, what findings have been achieved, what products have been created to tackle the same problem, what
    tools/techniques/approaches have been proven to be useful. The purpose is to help you to analyse the problem and its context, learn from existing related work, and
    propose appropriate solutions for your own project. You need to look for important concepts and principles, possible techniques, methods, tools, structures and etc that
    you could use in your project.
    Note that the marking scheme says that a ‘critical review’ is required. You should not simply be relaying information or paraphrasing the sources that you have found, but
    identifying principles, structuring ideas, discussing the relevance to your project, and making evaluative comments about what you have found. It is useful to end the
    literature review by summarising the key ideas that it has contributed to the project.

    Another pitfall to avoid is attempting to write about too many areas. You may find that there is a great deal of relevant literature and many subjects to review. It is
    better to aim for depth in the areas that are absolutely necessary and relevant to your project. An opposite example is the review where many areas are tackled but nothing
    of significance is said about any of them, and perhaps only one source (or no sources!) is cited for each. In this case, assuming that the project topic is suitable, the
    student would have been better advised to take a narrower focus.

    The length of literature review should be guided by the needs of the specific project, but they are typically 2500 – 3000 words. You should use respectable academic sources,
    such as refereed journals or conferences, or other reliable and authoritative sources of information.

    Once you have described your problem and put it into context by carrying out the literature review, you are in a position to identify and justify the appropriate approach
    to take to solving the problem. You will need to explain your approach at an overview level and give your reasons for choosing this approach. We are looking for a
    high-level discussion here. You will go into the details of exactly what you did (e.g. how you built the product, set up experiments, chose your participants, ran your
    tests, etc.) in the practical work chapters.

    If your practical work is to conduct an investigation into a research question / hypothesis, you should discuss the possible strategies for conducting research or
    investigations in your subject area. This could be quite different for projects in different subject areas, for example, forensics, networking or HCI. You may need to
    consider research methods – for example, are you using one of the many varieties of experimental design? are you carrying out your study in lab conditions, or in the
    field? will you be taking measurements of the performance of real equipment, or using a computer simulation? You should review the possible alternative approaches, and
    explain why your chosen strategy was preferable to others, or was the only possibility.

    If you plan to build a product, you need to identify the requirements for the product and produce its specification. The actual specification of requirements is part
    of your product and is marked under the product component. The research and planning chapters include your discussion and justification of those requirements. Reasons
    for choosing particular requirements could relate to user needs, constraints on project scope, to the findings of your literature review, etc. – whatever is appropriate
    to your particular project. You should discuss the methods you used to establish the product requirements, e.g., requirements elicitation and modelling, interviews,
    examination of existing products. It is appropriate to comment on significant findings that affect the decisions and to include any artefacts produced and the outputs
    of your investigations in the product documentation. It is expected that your requirements specification covers both functional and non-functional requirements. It is
    expected that your requirements specification covers both functional and non-functional requirements.

    You must also identify and justify the methods, tools, and techniques that you will use to conduct your investigation and/or build your product. Again, the decisions
    should be based on literature review and/or examination of related work. You should justify that the chosen tools and methods are appropriate for solving the problem
    and meeting the requirements, and explain the deliverables that you are going to produce.
    You may find that you now have several possible approaches and technologies that could be used to solve the problem. If this is the case, you should provide a short
    analysis of the possibilities justifying your selected route if the factors influencing your choices are worth discussing. There is no point in going through a spurious
    decision process if there is no real decision to be made. Many students choose a programming language or database or other software tool because they are familiar with it.
    As long as the tool is suitable, your own familiarity with it is sufficient reason for using that technology, and it is sufficient to indicate briefly why it is appropriate.
    If your project requires the use of some very specialised software or programming language, it may be useful to describe its main features as well as justifying its use,
    and if the focus of your project is exploring a particular tool or tool features, you will of course need to discuss the features in some depth. It is also suggested
    to briefly discuss the alternatives if the chosen tools/techniques do not work as expected. For some projects, there will not be any realistic alternatives to the tools
    and techniques that you are using. This is not a problem as long as you can justify it.
}
